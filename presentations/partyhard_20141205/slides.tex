
\documentclass{beamer}
\usepackage[utf8x]{inputenc}
\usepackage[russian]{babel}
\usepackage{mathtools}

\usetheme{Warsaw}
\title[hackerspace.by]
{Минский Хакерспейс - Открытая Лаборатория Технического Творчества}
\author{Алексей Гутиков}
\date{\today}

\newif\ifplacelogo % create a new conditional
\placelogofalse % do not place logo on title page (there is usual image)
\logo{\ifplacelogo\includegraphics[width=0.1\textwidth]{logo}\fi} % replace with your own command

\begin {document}

\titlegraphic{
	\includegraphics[width=0.3\textwidth]{logo}
}

\begin{frame}
\titlepage
\begin{center}
\end{center}
\end{frame}

\placelogotrue % place logo on every other page

\begin {frame}
\frametitle {Что такое хакерспейс}
\framesubtitle {Клуб технического творчества}
Место куда можно придти в любое время с друзьями, знакомыми и незнакомыми и устроить хакатон. 

При этом формат и тематика проектов ограничивается только геометрией помещения.
\end {frame}


\begin {frame}
\frametitle {Что такое хакерспейс}
\framesubtitle {Хостинг для проектов}
\begin{itemize}
\item софтовых
\item электронных
\item конструкторских
\item социальных
\item научных
\item художественных
\end{itemize}
\end {frame}


\begin {frame}
\frametitle {Чем он может и должен быть}
\framesubtitle {Многопрофильная лаборатория}
Центр реализации инженерных идей, 
клуб разработки каких-то новых, полезных, оригинальных устройств, приборов.
\end {frame}


\begin {frame}
\frametitle {Чем он может и должен быть}
\framesubtitle {Платформа для самообразовния}
В области:
\begin{itemize}
\item автоматического управления, 
\item электроники, 
\item механики, 
\end{itemize}
Где:
\begin{itemize}
\item процесс обучения проходит в основном через практику,
\item человек имея базовые знания и навыки, при помощи сообщества, сможет постичь новые знания и приобрести навыки.
\end{itemize}
\end {frame}


\begin {frame}
\frametitle {Зачем это всё на самом деле?}
\begin{itemize}
\item Не знаю с чего начать
\item Спросить, научиться
\item Живое общение с себе-подобными
\item Плавно переходящее в хакинг (работу над проектом)
\item Похвастаться
\end{itemize}
\end {frame}


\begin {frame}
\frametitle {Что у нас есть}
\framesubtitle {Сообщество}
\begin{itemize}
\item ~20 постоянных участников переменной активности
\item >100 сочувствующих в соцсетях и рассылке
\item
  \begin{itemize}
    \item \texttt{www.facebook.com/hs.minsk}
    \item \texttt{vk.com/hackerspace\_minsk}
    \item \texttt{groups.google.com/forum/\#!forum/hackerspace-minsk}
  \end{itemize}
\item Обмен опытом и знаниями
\end{itemize}
\end {frame}


\begin {frame}
\frametitle {Что у нас есть}
\framesubtitle {Помещение}
\begin{itemize}
\item 50 \(m^2\)
\item доступ \(24\times7\)
\item 220, отопление, освещение, приточно-вытяжная вентиляция, проточная вода, интернет, статический IP
\end{itemize}
\begin{center}
  \includegraphics[height=0.7\textheight]{hackerspace_view.jpg}
\end{center}
\end {frame}


\begin {frame}
\frametitle {Что у нас есть}
\framesubtitle {Инструменты}
\begin{itemize}
\item Пайка
\item Монтаж
\item Ручная обработка материалов
\end{itemize}
\end {frame}


\begin {frame}
\frametitle {Что у нас есть}
\framesubtitle {Проекты}
\begin{columns}
  \column{0.5\textwidth}
\begin{itemize}
\item 3D печать
\item ЧПУ станки вообще
\item Дроны, роботы
\item hackerspace.by/projects
\item ...
\end{itemize}
  \column{0.5\textwidth}
  \includegraphics[width=0.9\textwidth]{f.jpg}
\end{columns}
\end {frame}


\begin {frame}
\frametitle {Что мы хотим добавить}
\framesubtitle {Больше инструментов, оборудования и материалов}
\begin{itemize}
\item 3D печать
\item Полноценные рабочие места для обработки материалов
\item Сверлильный станок, циркулярка, ЧПУ фрезер, laser-cutter, точечная сварка, ...
\item прототипирование печатных плат
\item наборы радиоэлектронных компонентов, запас крепежных метизов, листовых материалов
\item локальный сервер
\item ...
\end{itemize}
\end {frame}


\begin {frame}
\frametitle {Что мы хотим добавить}
\framesubtitle {Больше драйва}
Больше оборудования -> больше возможностей -> 

проект быстрее доходит до стадии рабочего прототипа.
\end {frame}


\begin {frame}
\frametitle {Что мы хотим добавить}
\framesubtitle {Больше коммуникации}
Социальная сеть для хакерспейса.
\end {frame}


\begin {frame}
\begin{center}
СПАСИБО ЗА ВНИМАНИЕ

Заходите

\texttt{hackerspace.by}
\end{center}
\end {frame}

\end {document}
